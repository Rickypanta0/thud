%% Le lingue utilizzate, che verranno passate come opzioni al pacchetto babel. Come sempre, l'ultima indicata sarà quella primaria.
%% Se si utilizzano una o più lingue diverse da "italian" o "english", leggere le istruzioni in fondo.
\def\thudbabelopt{english}
%% Valori ammessi per target: bach (tesi triennale), mst (tesi magistrale), phd (tesi di dottorato).
%% Valori ammessi per aauheader: '' (vuoto -> nessun header Alpen Adria Univeristat), aics (Department of Artificial Intelligence and Cybersecurity), informatics (Department of Informatics Systems). Il nome del dipartimento è allineato con la versione inglese del logo UniUD.
%% Valori ammessi per style: '' (vuoto -> stile moderno), old (stile tradizionale).
\documentclass[target=bach,aauheader=,style=]{thud}

%% --- Informazioni sulla tesi ---
%% Per tutti i tipi di tesi
% Scommentare quello di interesse, o mettete quello che vi pare
\course{Informatica}
%\course{Internet of Things, Big Data e Web}
%\course{Matematica}
%\course{Comunicazione Multimediale e Tecnologie dell'Informazione}
\title{Automatic Nuclei Segmentation and Counting in Histopathology Images with U-Net}
\author{Pantanali Riccardo}
\supervisor{Prof.\ Della Mea Vincenzo}
\cosupervisor{Dott.ssa \ Teresa Pace}
\tutor{Guido Necchi}
%% Campi obbligatori: \title, \author e \course.
%% Altri campi disponibili: \reviewer, \tutor, \chair, \date (anno accademico, calcolato in automatico), \rights
%% Con \supervisor, \cosupervisor, \reviewer e \tutor si possono indicare più nomi separati da \and.
%% Per le sole tesi di dottorato:
\phdnumber{313}
\cycle{XXVIII}
\contacts{Via della Sintassi Astratta, 0/1\\65536 Gigatera --- Italia\\+39 0123 456789\\\texttt{http://www.example.com}\\\texttt{inbox@example.com}}

%% --- Pacchetti consigliati ---
%% pdfx: per generare il PDF/A per l'archiviazione. Necessario solo per la versione finale
\usepackage[a-1b]{pdfx}
%% hyperref: Regola le impostazioni della creazione del PDF... più tante altre cose. Ricordarsi di usare l'opzione pdfa.
\usepackage[pdfa]{hyperref}
%% tocbibind: Inserisce nell'indice anche la lista delle figure, la bibliografia, ecc.

%% --- Stili di pagina disponibili (comando \pagestyle) ---
%% sfbig (predefinito): Apertura delle parti e dei capitoli col numero grande; titoli delle parti e dei capitoli e intestazioni di pagina in sans serif.
%% big: Come "sfbig", solo serif.
%% plain: Apertura delle parti e dei capitoli tradizionali di LaTeX; intestazioni di pagina come "big".

\begin{document}
\maketitle

%% Dedica (opzionale)
\begin{dedication}
	Al mio cane,\par per avermi ascoltato mentre ripassavo le lezioni.
\end{dedication}

%% Ringraziamenti (opzionali)
\acknowledgements
Sed vel lorem a arcu faucibus aliquet eu semper tortor. Aliquam dolor lacus, semper vitae ligula sed, blandit iaculis leo. Nam pharetra lobortis leo nec auctor. Pellentesque habitant morbi tristique senectus et netus et malesuada fames ac turpis egestas. Fusce ac risus pulvinar, congue eros non, interdum metus. Mauris tincidunt neque et aliquam imperdiet. Aenean ac tellus id nibh pellentesque pulvinar ut eu lacus. Proin tempor facilisis tortor, et hendrerit purus commodo laoreet. Quisque sed augue id ligula consectetur adipiscing. Vestibulum libero metus, lacinia ac vestibulum eu, varius non arcu. Nam et gravida velit.

%% Sommario (opzionale)
\abstract
Nunc ac dignissim ipsum, quis pulvinar elit. Mauris congue nec leo ornare lobortis. Nulla hendrerit pretium diam nec lobortis. Nullam aliquam laoreet nisl, sit amet facilisis lectus accumsan ut. Duis et elit hendrerit metus venenatis condimentum. Integer id eros molestie, interdum leo sit amet, aliquet metus. Integer fermentum tristique magna, vel luctus neque rhoncus vel. Ut hendrerit et quam et semper. Mauris egestas, odio sed aliquet luctus, magna orci euismod odio, vitae lacinia tellus tellus non lectus. Aliquam urna neque, porta et mattis aliquam, congue sit amet lorem. In ultrices augue sit amet ante vehicula, vitae rhoncus turpis auctor. Donec porta scelerisque eros, at mollis enim imperdiet ut. 

%% Indice
\tableofcontents

%% Lista delle tabelle (se presenti)
%\listoftables

%% Lista delle figure (se presenti)
%\listoffigures

%% Corpo principale del documento
\mainmatter

%% Parte
%% La suddivisione in parti è opzionale; solitamente sono sufficienti i capitoli.
%\part{Parte}

%% Capitolo
\chapter{Dataset}
The analysis of nuclear features in histopathology images has long been recognized as a crucial aspect of cancer research and diagnosis. The part that seems like a bottleneck is the limited size of the most publicy available dataset for nuclei segmentation and classification, and often suffering from sparse labelling or sampling bias.\\
To address this limitations, Gamper et al. (2019) 
\cite{gamper2019pannuke} introduce PanNuke, an open pan-cancer a semi-automatically obtained for nuclei instance segmentation and classification dataset.

%% Sezione
\section{Materials}
PanNuke is created through multiple model combinantions with already existing public data for nuclei classification$/$segmentation. To produce the proposed dataset have been sampled patches randomly extracted from a Whole Slide Image (WSI), using 19 different TCGA tissue types and other internal dataset for prostate, colon, ovarian, breast and oral tissue.\\
In total, 455 visual field were collected, of which 312 were randomly sampled from more than 200.000 H\&E-stained WSI. To overcome the scarity of ground truth and the unreliability of manual labeling, have been adopted a three stream approach regarding segmentation, classification and detection models.\\
For the segmentation stream, the authors used the Kumar \cite{article} dataset and CPM17 \cite{vu2018methodssegmentationclassificationdigital} dataset. For the detection stream, additional images were extracted from TCGA and the Bone Morrow \cite{10.1007/978-3-319-24574-4_33} dataset  . The classification stream, the nuclei were labeled using MonuSeg, ColonNuclei, SPIE cellularity challenge data, and the Nuclei Attribute dataset, ensuring a diverse set of nuclear categories.
\section{Method}
The authors followed three pathways to abtaind the validated dataset. These three pathways are segmentation, detection and classification. The mainly reason because they choose this type of approach, is to evaluate the performance of each type in a more realistc environment.\\
The models
\section{Class distribution}




%% Fine dei capitoli normali, inizio dei capitoli-appendice (opzionali)
\appendix

%\part{Appendici}

\chapter{Titolo della prima appendice}
Sed purus libero, vestibulum ut nibh vitae, mollis ultricies augue. Pellentesque velit libero, tempor sed pulvinar non, fermentum eu leo. Duis posuere eleifend nulla eget sagittis. Nam laoreet accumsan rutrum. Interdum et malesuada fames ac ante ipsum primis in faucibus. Curabitur eget libero quis leo porttitor vehicula eget nec odio. Proin euismod interdum ligula non ultricies. Maecenas sit amet accumsan sapien.

%% Parte conclusiva del documento; tipicamente per riassunto, bibliografia e/o indice analitico.
\backmatter

%% Riassunto (opzionale)
%\summary
%Maecenas tempor elit sed arcu commodo, dapibus sagittis leo egestas. Praesent at ultrices urna. Integer et nibh in augue mollis facilisis sit amet eget magna. Fusce at porttitor sapien. Phasellus imperdiet, felis et molestie vulputate, mauris sapien tincidunt justo, in lacinia velit nisi nec ipsum. Duis elementum pharetra lorem, ut pellentesque nulla congue et. Sed eu venenatis tellus, pharetra cursus felis. Sed et luctus nunc. Aenean commodo, neque a aliquam bibendum, mauris augue fringilla justo, et scelerisque odio mi sit amet diam. Nulla at placerat nibh, nec rutrum urna. Donec ut egestas magna. Aliquam erat volutpat. Phasellus vestibulum justo sed purus mattis, vitae lacinia magna viverra. Nulla rutrum diam dui, vel semper mi mattis ac. Vestibulum ante ipsum primis in faucibus orci luctus et ultrices posuere cubilia Curae; Donec id vestibulum lectus, eget tristique est.

%% Bibliografia (praticamente obbligatoria)
\bibliographystyle{plain_\languagename}%% Carica l'omonimo file .bst, dove \languagename è la lingua attiva.
%% Nel caso in cui si usi un file .bib (consigliato)
\bibliography{thud}
%% Nel caso di bibliografia manuale, usare l'environment thebibliography.
%% Per l'indice analitico, usare il pacchetto makeidx (o analogo).

\end{document}

--- Istruzioni per l'aggiunta di nuove lingue ---
Per ogni nuova lingua utilizzata aggiungere nel preambolo il seguente spezzone:
    \addto\captionsitalian{%
        \def\abstractname{Sommario}%
        \def\acknowledgementsname{Ringraziamenti}%
        \def\authorcontactsname{Contatti dell'autore}%
        \def\candidatename{Candidato}%
        \def\chairname{Direttore}%
        \def\conclusionsname{Conclusioni}%
        \def\cosupervisorname{Co-relatore}%
        \def\cosupervisorsname{Co-relatori}%
        \def\cyclename{Ciclo}%
        \def\datename{Anno accademico}%
        \def\indexname{Indice analitico}%
        \def\institutecontactsname{Contatti dell'Istituto}%
        \def\introductionname{Introduzione}%
        \def\prefacename{Prefazione}%
        \def\reviewername{Controrelatore}%
        \def\reviewersname{Controrelatori}%
        %% Anno accademico
        \def\shortdatename{A.A.}%
        \def\summaryname{Riassunto}%
        \def\supervisorname{Relatore}%
        \def\supervisorsname{Relatori}%
        \def\thesisname{Tesi di \expandafter\ifcase\csname thud@target\endcsname Laurea\or Laurea Triennale\or Dottorato\fi}%
        \def\tutorname{Tutor aziendale}%
        \def\tutorsname{Tutor aziendali}%
    }
sostituendo a "italian" (nella 1a riga) il nome della lingua e traducendo le varie voci.
